\section{Lojban grammar}
\begin{frame}{Bridi}
    \textbf{Bridi} is the most central unit of Lojban utterances. The concept is very close to what we call a \textbf{proposition} in English.

    \pause
    A bridi is a \textbf{claim} that some objects stand in a \textbf{relation} to each other, or that an object has some \textbf{property}.
\end{frame}

\begin{frame}{Jufra}
    This stands in contrast to \textbf{jufra}, which are merely Lojban \textbf{utterances}, which can be bridi or anything else being said.
    % Utterance translates to "espressione" in italian

    \pause
    Since they always state something, bridi, are either \textbf{true or false}, while not all jufra can be said to be such.
\end{frame}

\begin{frame}{Bridi or jufra?}
    ``Mozart was the greatest musician of all time'' $\rightarrow$
    \pause
    bridi

    \pause
    ``Ouch! My toe!''\\$\rightarrow$
    \pause
    jufra
\end{frame}

\begin{frame}{Bridi components}
    A bridi consists of one \textbf{selbri}, and one or more \textbf{sumti}.

    \pause
    The selbri is the \textbf{relation} or claim about the object, and the sumti are the \textbf{objects} which are in a relation.
    \pause
    % Spectators should write this one down
    \begin{align*}
        \overbrace{\underbrace{\text{mi}}_{\text{sumti}}~\underbrace{\text{nelci}}_{\text{selbri}}~\underbrace{\text{do}}_{\text{sumti}}}^{\text{bridi}}
    \end{align*}
\end{frame}

\begin{frame}{Building simple bridi}
    Let's try making Lojban bridi. We'll need some words, which can act as sumti (objects):
    \begin{itemize}
        \item \textbf{mi} = ``I'' or ``we''
        % Singular and plural
        \item \textbf{do} = ``You''
        % Something you can point to
        \item \textbf{ti} = ``this''
    \end{itemize}
\end{frame}

\begin{frame}{Building simple bridi II}
    We'll also need some words, which can act as selbri (relations):
    \begin{itemize}
        % This would be a verb
        \item \textbf{dunda} = $x_1$ gives $x_2$ to $x_3$ (without payment)
        % This would be an adjective
        \item \textbf{pelxu} = $x_1$ is yellow
        % This would be a noun
        \item \textbf{zdani} = $x_1$ is a home of $x_2$
    \end{itemize}
\end{frame}

\begin{frame}{Building simple bridi III}
    Selbri translations have some \textbf{sumti placeholders}: \(x_1, \ldots, x_n\). A bridi is constructer concatenating $x_1$ with the selbri and any other sumti.

    To say ``I give this to you'' you just say: ``\underline{mi dunda ti do}''.

    Multiple bridi after each other are separated by \textbf{.i} This is the Lojban equivalent of \textbf{full stop}. It goes before bridi but is usually implicit.
\end{frame}

\begin{frame}{Build your own bridi}
    Using the selbri and sumti we saw:
    \begin{itemize}
        \item \textbf{pelxu} = $x_1$ is yellow
        \item \textbf{dunda} = $x_1$ gives $x_2$ to $x_3$ (without payment)
    \end{itemize}

    Try to translate these english propositions:
    \begin{itemize}
        \item<2-> ``This is yellow''
        \item<3-> ``You give this to me''
    \end{itemize}
\end{frame}

\begin{frame}{The empty sumti}
    Most selbri have from one to five sumti places, here's one with four sumti places:
    \begin{itemize}
        \item \textbf{vecnu} = $x_1$ sells $x_2$ to $x_3$ for price $x_4$
    \end{itemize}

    \pause
    If I want to say ``I sell this'', it would be too much to have to fill the sumti places $x_3$ and $x_4$.
\end{frame}

\begin{frame}{The empty sumti II}
    We can use \textbf{zo\textquotesingle e} each time we don't want to specify one or more sumti places or when they can be determined from context.

    \pause
    So to say ``I sell to you'', I could say:
    \begin{center}
        \textbf{mi vecnu zo\textquotesingle e do zo\textquotesingle e}
    \end{center}
\end{frame}

\begin{frame}{Omitting sumti places}
    Still, filling out three zo'e just to say that a thing is being sold \textbf{takes time}.

    \pause
    We can \textbf{omit} sumti places at both \textbf{edges} of a bridi and they will be parsed as zo\textquotesingle e.
\end{frame}

\begin{frame}{Selbri positioning}
    The form bridi structure we already saw,
    \[x_1, selbri, x_2,\ldots,x_n,\]
    is \textbf{not mandatory}.

    You can put the selbri everywhere you want, except for the beginning. Otherwise, the $x_1$ would be \textbf{left out} and filled with zo\textquotesingle e.
\end{frame}

\begin{frame}{Reordering sumti places}
    Lojban grammar rules allow us to \textbf{reorder} the sumti in a phrase by means of \textbf{fa}, \textbf{fe}, \textbf{fi}, \textbf{fo} and \textbf{fu}.

    Those words respectively \textbf{tag the next sumti} as $x_1$, $x_2$, $x_3$, $x_4$, $x_5$. To use an example:
    \begin{center}
        \textbf{vecnu fe ti}
    \end{center}
    Translates to ``this is being sold''.
\end{frame}

\begin{frame}{Selbri to sumti conversion}
    We still have a big problem left, what if we wanted to say that we're selling to a German person?

    \begin{itemize}
        \item \textbf{dotco} = $x_1$ is German in aspect $x_2$
    \end{itemize}

    We cannot say ``mi vecnu zo\textquotesingle e dotco'' since we would have \textbf{two selbri in a bridi}.
\end{frame}

\begin{frame}{Selbri to sumti conversion II}
    We will two special words:
    % These are basically parentheses
    \begin{itemize}
        \item \textbf{lo} = begin selbri to sumti conversion
        \item \textbf{ku} = end conversion
    \end{itemize}

    This way we could say ``\underline{lo zdani ku}'' meaning one or more houses for someone.
    % Translating "I'm selling something to a German" is left as an exercise to the reader
\end{frame}

\begin{frame}{Bridi to selbri conversion}
    We can apply a \textbf{similar conversion} for cases in which we want to specify, for instance, ``I'm happy that you are my friend''.

    \begin{itemize}
        \item \textbf{gleki} = $x_1$ is happy about $x_2$
        \item \textbf{pendo} = $x_1$ is a friend to $x_2$
    \end{itemize}
\end{frame}

\begin{frame}{Bridi to selbri conversion II}
    We're gonna use the (generic) bridi conversion word:

    \begin{itemize}
        \item \textbf{su\textquotesingle u} = $x_1$ is an abstraction of bridi of type $x_2$
        \item \textbf{kei} = end abstraction
    \end{itemize}

    \pause
    The proper translation is:
    \begin{center}
        \textbf{mi gleki lo su\textquotesingle u do pendo mi kei ku}
    \end{center}
\end{frame}
