\section{Lojban grammar}
\begin{frame}{Bridi}
    \textbf{Bridi} is the most central unit of Lojban utterances. The concept is very close to what we call a \textbf{proposition} in English.

    \pause
    A bridi is a \textbf{claim} that some objects stand in a \textbf{relation} to each other, or that an object has some \textbf{property}.
\end{frame}

\begin{frame}{Jufra}
    This stands in contrast to \textbf{jufra}, which are merely Lojban \textbf{utterances}, which can be bridi or anything else being said.
    % Utterance translates to "espressione" in italian

    \pause
    Since they always state something, bridi, are either \textbf{true or false}, while not all jufra can be said to be such.
\end{frame}

\begin{frame}{Bridi or jufra?}
    ``Mozart was the greatest musician of all time'' $\rightarrow$
    \pause
    bridi

    \pause
    ``Ouch! My toe!''\\$\rightarrow$
    \pause
    jufra
\end{frame}

\begin{frame}{Bridi components}
    A bridi consists of one \textbf{selbri}, and one or more \textbf{sumti}.

    \pause
    The selbri is the \textbf{relation} or claim about the object, and the sumti are the \textbf{objects} which are in a relation.
    \pause
    % Spectators should write this one down
    \begin{align*}
        \overbrace{\underbrace{\text{mi}}_{\text{sumti}}~\underbrace{\text{nelci}}_{\text{selbri}}~\underbrace{\text{do}}_{\text{sumti}}}^{\text{bridi}}
    \end{align*}
\end{frame}

\begin{frame}{Building simple bridi}
    Let's try making Lojban bridi. We'll need some words, which can act as sumti (objects):
    \begin{itemize}
        \item \textbf{mi} = ``I'' or ``we''
        % Singular and plural
        \item \textbf{do} = ``You''
        % Something you can point to
        \item \textbf{ti} = ``this''
    \end{itemize}
\end{frame}

\begin{frame}{Building simple bridi II}
    We'll also need some words, which can act as selbri (relations):
    \begin{itemize}
        % This would be a verb
        \item \textbf{dunda} = $x_1$ gives $x_2$ to $x_3$ (without payment)
        % This would be an adjective
        \item \textbf{pelxu} = $x_1$ is yellow
        % This would be a noun
        \item \textbf{zdani} = $x_1$ is a home of $x_2$
    \end{itemize}
\end{frame}

\begin{frame}{Building simple bridi III}
    Selbri translations have some \textbf{sumti placeholders}: \(x_1, \ldots, x_n\). A bridi is constructer concatenating $x_1$ with the selbri and any other sumti.

    To say ``I give this to you'' you just say: ``\underline{mi dunda ti do}''.

    Multiple bridi after each other are separated by \textbf{.i} This is the Lojban equivalent of \textbf{full stop}. It goes before bridi but is usually implicit.
\end{frame}

\begin{frame}{Build your own bridi}
    Using the selbri and sumti we saw:
    \begin{itemize}
        \item \textbf{pelxu} = $x_1$ is yellow
        \item \textbf{dunda} = $x_1$ gives $x_2$ to $x_3$ (without payment)
    \end{itemize}

    Try to translate these english propositions:
    \begin{itemize}
        \item<2-> ``This is yellow''
        \item<3-> ``You give this to me''
    \end{itemize}
\end{frame}
