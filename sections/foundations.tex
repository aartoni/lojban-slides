\section{Lojban foundations}
\begin{frame}{Pronouncing Lojban}
    The Lojban alphabet has six vowels and eighteen consonants.
\end{frame}

\begin{frame}{Lojban vowels}
    These are pretty much the same as vowels in Italian or Spanish.
    \uncover<2->{\begin{multicols*}{2}
        \begin{itemize}
            \item a $\rightarrow$ a
            \item e $\rightarrow$ e
            \item i $\rightarrow$ i
            \item o $\rightarrow$ o
            \item u $\rightarrow$ u
            \item<3> y $\rightarrow$ ə
        \end{itemize}
    \end{multicols*}}
\end{frame}

\begin{frame}{Lojban consonants}
    The Lojban consonants are the same as the English, except that Lojban doesn't use the letters \textbf{H}, \textbf{Q} or \textbf{W}.

    Most of the consonants are pronounced like in English, but there are some exceptions.

    \uncover<2->{\begin{multicols*}{2}
        \begin{itemize}
            \item<2-> g $\rightarrow$ \underline{g}um
            \item<3-> c $\rightarrow$ \underline{sh}ip
            \item<4-> j $\rightarrow$ bon\underline{j}our
            \item<5-> x $\rightarrow$ Ba\underline{ch}
            % R is usually rolled
        \end{itemize}
    \end{multicols*}}
\end{frame}

\begin{frame}{The semi-consonant}
    Even though Lojban doesn't use the letter H there's a \textbf{(semi-)consonant}, which is the \textbf{apostrophe}.

    \uncover<2->{It is only \textbf{used between two vowels} to prevent them from running into each other.}

    \uncover<3>{Thus \underline{ui} is normally pronounced "we", but \underline{u\textquotesingle i} is "oohee".}
\end{frame}

\begin{frame}{Writing Lojban}
    We put full stops (periods) in front of any word starting with a vowel, they represent \textbf{glottal stops} (short pauses).
    % The consonant in the middle of "ah-ah".

    We do this because Lojban words usually ends in vowels.
\end{frame}

\begin{frame}{Writing Lojban II}
    We only use lowercase letters and we avoid every other kind of punctuation marks.

    We avoid things like question marks and parenthesis because Lojban already has words for expressing them.
    % Usually lojbanists think adding exotic punctuation creates an unwanted difference between written and spoken language.
\end{frame}

\begin{frame}
    \centering
    \Large
    Are you following me?

    \pause
    \textbf{go\textquotesingle i}
\end{frame}

\begin{frame}{Translate a cmevla (proper name)}
    \begin{itemize}
        \item \textbf{Richard Nixon} \uncover<2->{$\Rightarrow$ \textbf{.ritcyrd.niksyn.}}
        \item<3-> \textbf{William Shakespeare} \uncover<4->{$\Rightarrow$ \textbf{.uiliam.cekspir.}}
    \end{itemize}
\end{frame}
